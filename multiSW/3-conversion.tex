\section{Comparison of  \emph{lemon} and UBY-LMF}
%\begin{verbatim}
%- modeling issues: argument mapping to capture syntactic behavior -> JEK&JM
%- usage/register information represented differently
%- lemonUBYlite: where, which license
%\end{verbatim}
%The actual conversion  of UBY lexica to \emph{lemon} was achieved by means of
%an XML style sheet transform~\footnote{The XSL file can be downloaded at
%\url{https://raw.github.com/jmccrae/lemon.api/master/src/main/resources/xslt/ubylmf2lemon.xsl}}.
%The following UBY lexica were converted: WordNet, FrameNet, VerbNet, English and German Wiktionary, and the English and German entries of OmegaWiki. 
%Pairs of these resources are linked at the sense level:
%there are monolingual links between 
%VerbNet--FrameNet\footnote{\url{http://verbs.colorado.edu/semlink/}}, 
%VerbNet--WordNet\footnote{\url{http://verbs.colorado.edu/~mpalmer/projects/verbnet}},
%WordNet--FrameNet \cite{tonelli-pighin:2009:CoNLL,laparra-rigau:2009:RANLP09}, as well as between WordNet--Wiktionary  \cite{meyer2011what}. Moreover, \emph{lemonUby} provides cross-lingual links between WordNet and the German Ome\-ga\-Wiki \cite{gurevych2012uby}, also including the inter-language links already given in OmegaWiki.

%The resulting resource 
%\emph{lemonUby} has been published at
%\url{http://lemon-model.net/lexica/uby}
%and is available under an open CC-BY-SA license. 
%The choice of a share-alike
%license was due to UBY being published under the same license. Statistics for
%the resources are given in table \ref{triples}\footnote{At the time of
%submission, en.wiktionary.org resource was not available, we expect to have
%this ready in time for final publication}.
%Based on the mapping of UBY-LMF to lemon, we performed a conversion of the following
%LRs which demonstrate the main parts of a UBY-LMF -- lemon
%mapping: WordNet, FrameNet, VerbNet, and multilingual OmegaWiki. 
%These LRs illustrate a number of differences between the two lexicon models in a 
%prototypical way. First, we will describe differences in the representation of senses, 
%then we will summarize more general differences that
%had to be harmonized via a mapping. The actual mapping was achieved by means of
%an XML stylesheet transform~\footnote{The XSL file can be downloaded at
%\url{https://raw.github.com/jmccrae/lemon.api/master/src/main/resources/xslt/ubylmf2lemon.xsl}}.

\emph{lemonUby} nicely illustrates how the two lexicon models represent senses in different ways:
\begin{itemize}
\item In line with previous authors~\cite{mccrae2012integrating}, synsets in
  WordNet and OmegaWiki are considered as ontology classes. Hypernym
  relationships stated in the lexicon are mapped directly to subclass
  relationships in the ontology. In UBY-LMF, Synset is part of the lexicon.
%\item Synsets, which are used in WordNet and OmegaWiki, are considered as
%  entryless senses in \emph{lemon}. Previous
%  conversions~\cite{mccrae2012integrating} have treated synsets as ontological
%  classes, however here the 
\item The semantic predicate in UBY-LMF is used to represent semantic frames from 
  FrameNet \cite{TUD-CS-2013-0003}; frames consist of senses which evoke the same situation with participants. 
  Thus, senses in the same semantic frame are semantically related, but not synonymous; 
  e.g., the verbs {\em love} and {\em hate} are both in the same frame. %Experiencer subj frame
In \emph{lemon} a 
{\em broader sense} is used to capture semantic predicates.
\item VerbNet verb classes and their hierarchical organization correspond to  
  subcat[egorization] frame set in UBY-LMF. VerbNet classes group verbs that
share the same syntactic subcategorization frame, semantic roles, selectional
restrictions, and semantic predicate. Although the resulting verb classes are
semantically coherent, the semantic relatedness of verb senses in a VerbNet class is much more
distant that in FrameNet, e.g., the verbs {\em believe}, {\em swear} and {\em doubt} are in the same class.
%conjecture-29.5-1.
%semantically related and 
%  syntactically similar senses; the semantic relatedness of senses in one VerbNet class is much more
%distant that in FrameNet. 
Verb classes in \emph{lemon} are represented by a broader sense as well.
The hierarchic organisation of VerbNet classes is thus lost, but this can be
reconstructed by means of an axiomatic description of subcategorization frames in
order to create a hierarchy similar as in the LexInfo linguistic
ontology~\cite{mccrae2011linking}.
\item Syntactic behaviour in UBY-LMF mainly provides information on
subcategorization frames, which are specified by means of syntactic arguments
with rich morpho-syntactic information attached
\cite{TUD-CS-2012-0024}.
SyntacticBehaviour is merely a property instead of a node in \emph{lemon}: in 
contrast to LMF, there is no need for an explicit link between senses and
subcategorization frames. Instead this information  can be 
reconstructed from the \emph{synArg} - \emph{semArg} mapping.
Lexica which do not specify semantic arguments, e.g., GermaNet \cite{Kunze02},
must provide a specific annotation to allow this information to be represeneted.
\item Translations are represented differently. UBY-LMF uses Equivalent for 
  translations that are not sense-disambiguated \cite{TC32012}, where as
  \emph{lemon} requires that this link is made at the sense level. In the case
  where it was not possible to infer the sense, the lexical entries a `stub'
  lexical entry with a single sense.\footnote{UBY-LMF can also represent sense-disambiguated 
    translations by means of a sense axis which connects senses from different
  lexica.}
\end{itemize}
%To sum up, 
Most of the differences between UBY-LMF and \emph{lemon} derive from the fact that 
\emph{lemon} keeps the lexicon and ontology layers separate. In this way, sense representations in
\emph{lemon} are more compact compared to the more distributed sense representations in UBY-LMF.

%the separation of  the principal distinction between UBY-LMF and \emph{lemon}the separation of 
%a principal difference between UBY-LMF and \emph{lemon} is redundant vs. compact  representation of information.
%For instance, the synset class is merged into the lemon:LexicalSense, while UBY might provide information on
%synonyms by sense relations of type synonym and by the Synset class (which groups synonymous senses).
%Since UBY-LMF is populated fully-automatically, the consistency of redundant information can be ensured.
%An example is SyntacticBehaviour which is not used in \emph{lemon}: while the information to which sense a subcat frame belongs to, can be reconstructed from the \emph{synArg} - \emph{semArg} Mapping, the 
%relationship between sense and syntactic behaviour can not be recovered for LR which do not specify
%semantic arguments, e.g., GermaNet.

%\begin{table}
%  \begin{tabular}{|p{3cm}|c|}
%    \hline
%    Resource & Triples \\
%    \hline
%    WordNet & 5,102,744 \\
%    VerbNet & 570,256 \\
%    FrameNet & 1,110,763 \\
%    OmegaWiki English & 6,173,515 \\
%    OmegaWiki German & 5,310,551 \\
%    Wiktionary German & 4,766,917 \\
%    \hline
%    Total & 23,034,746 \\
%    \hline
%  \end{tabular}
%  \caption{Number of triples for each resource\label{triples}}
%\end{table}
    

\section{Converting UBY lexica to  \emph{lemon}: \emph{le\-mon\-U\-by}}

\begin{table}
  \begin{tabular}{|p{3cm}|cc|}
    \hline
    \emph{lemonUby} Resource & Triples & Links\\
    \hline
    WordNet & 5,102,744 & 196,420 \\
    VerbNet & 570,256 & 24,425\\
    FrameNet & 1,110,763 & 61,070 \\
    OmegaWiki English & 6,173,515 & 113,930\\
    OmegaWiki German & 5,310,551 & 159,978\\
    Wiktionary German & 4,766,917 & - \\
    Wiktionary English & 9,881,730 & 99,797\\
    \hline
  Total & 32,916,476 & 327,810 \\
    \hline
  \end{tabular}
  \caption{Number of triples and links for each resource.\label{triples}}
 \end{table}

\noindent 
To automatically convert data from UBY-LMF (extracted from UBY) to  \emph{lemon},
we performed a mapping of UBY-LMF elements to \emph{lemon} concepts and properties.
    % CC: from sect. 3
This involves two aspects: Formalizing and mapping UBY data structures in a Semantic-Web compliant way,
and the actual conversion of UBY data. 

\subsection{Modeling Data Categories: \emph{ubyCat}}

\noindent 
In UBY-LMF, most of the terms used and many features (e.g., the feature ``partOfSpeech'' and the corresponding attribute values)
are linked to % standardized %% CC: N�, ISOcat ist standardisiert, nicht der Inhalt
%% JEK Inhalt und Form lassen sich schwer trennen ;) - aber ich wei� was du meinst
community-maintained data categories in ISOcat.\footnote{\url{http://www.isocat.org/rest/dcs/484}}
As the mapping of UBY-LMF to \emph{lemon} preserves this linking, \emph{le\-mon\-U\-by}
is linked to ISOcat as well.
The content of ISOcat is also available as Linked Data \cite{ldl-windhouwer}, 
% actually, it isn't in the cloud diagram, for the reasons mentioned below -- and neither should be GOLD, for the same reasons. 
%But for GOLD, there is at least an implicit note on the website that the content provided should be available
and therefore, provides a possible and direct way to interconnect \emph{le\-mon\-U\-by} with other LLOD resources 
at the level of linguistic data categories.

% not relevant any more (ISOcat will have an open license)
% However, ISOcat does currently not contain an explicit license statement, even though its content is available over the internet.
% While people involved (Menzo Windhouwer, p.c., June 2012) believe that a publication under
% an open license would be appreciated, it has not been clarified whether the
% restrictive licensing policy of the ISO that applies to the ISOcat technical
% specifications also extends to the ISOcat data categories.
% A similar problem persists with the General Ontology of Linguistic Description \cite[GOLD]{farrar2003markup}
% %(GOLD)
% \footnote{\url{http://linguistics-ontology.org/}}, another LLOD-linked % this is true, at least from OLiA
% terminology repository.
%Currently, GOLD does not contain
%an explicit license statement as well, but it is available on-line, and as a community
%project,
%publication under an open license can be expected.

However, ISOcat is not a formal ontology, but only a semistructured collection of terms, and while it serves as a 
repository of definitions, it does not provide a formal data model that can be applied to a resource: 
ISOcat contains doublets created by different data providers, %e.g., VoiceProperty [DC-3551] vs. voice [DC-1433], 
and such superficially similar categories may actually have incompatible definitions, e.g., gerundive [DC-1294] is an ``adjective formed from a verb'' (excluding verbal nouns), whereas gerundive [DC-2243] is a ``non-finite form (...) other than the infinitive'' (including verbal nouns). Hierarchical relations between ISOcat terms are possible, but not obligatory, and when compared with a full-fledged ontology, ISOcat terms that represent superconcepts for a bundle of features 
%(e.g., ActiveVoice [DC-3064] dcif:isA VoiceProperty [DC-3551]) 
do not distinguish relational and categorial aspects.
%VoiceProperty could be either a property that assigns ActiveVoice to a particular unit of annotation, or a concept that defines the range of such a relation.

To render UBY/ISOcat data categories in a formal data model, we created the OWL/DL 
ontology \emph{ubyCat}\footnote{\url{http://purl.org/olia/ubyCat.owl}} which defines the semantics of
linguistic terms used in \emph{le\-mon\-U\-by}:
With respect to \emph{data structures}, it extends the \emph{lemon} ontology,\footnote{\url{http://www.monnet-project.eu/lemon}} concepts 
that are equivalent with \emph{lemon} but have a different label are included, flagged as deprecated (and not used during the conversion),
but left in the ontology to help UBY-LMF users find the equivalent \emph{lemon} categories. 

With respect to \emph{data categories}, \emph{ubyCat} is linked to the Ontologies of Linguistic Annotations
(OLiA) \cite{chiarcos2008ontology,chiarcos2012ontologies},\footnote{\url{http://purl.org/olia}} a modular 
architecture of OWL/DL ontologies that link resource-specific linguistic terminology (as for UBY) to an overarching
`Reference Model' which (a) provides a particular view on linguistic terminology, and (b) serves as an interface to
multiple community-maintained terminology repositories such as ISOcat with which it is linked. Details of this
linking are described in Sect.\ \ref{sec-olia}.

\subsection{Converting the Data}

\noindent
The actual conversion  of UBY data was achieved by means of
an XML style sheet transform\footnote{The XSL file can be downloaded at
\url{https://raw.github.com/jmccrae/lemon.api/master/src/main/resources/xslt/ubylmf2lemon.xsl}}
that implements a mapping of the UBY-LMF model to the \emph{lemon} model.

The following data from UBY were converted: WordNet, FrameNet, VerbNet,
English and German Wiktionary, the English and German
entries of Ome\-ga\-Wi\-ki,
as well as links between pairs of these lexicons at the word sense level:

\begin{itemize}
 \item monolingual manual sense links for
VerbNet--Frame\-Net,\footnote{\url{http://verbs.colorado.edu/semlink/}}
VerbNet--WordNet,\footnote{\url{http://verbs.colorado.edu/~mpalmer/projects/verbnet}} and 
automatic sense links for WordNet--Frame\-Net \cite{tonelli-pighin:2009:CoNLL,laparra-rigau:2009:RANLP09},
and WordNet--Wiktionary \cite{meyer2011what};
\item cross-lingual automatic sense links between WordNet and German Ome\-ga\-Wiki \cite{gurevych2012uby}, and manual inter-language links
already given in Ome\-ga\-Wi\-ki.
\end{itemize}
%% CC: slightly shortened

The resulting resource \emph{le\-mon\-U\-by} has been published at
\url{http://lemon-model.net/lexica/uby}
under an open CC-BY-SA license. The choice of a share-alike
license was due to UBY being published under the same license.
Statistics for
the \emph{le\-mon\-U\-by} resources including the sense links within this dataset
are given in table \ref{triples}. An obvious gap is the 
German Wiktionary which is currently not linked to any other resource within
\emph{le\-mon\-U\-by}. Future work on increasing the link density within \emph{le\-mon\-U\-by}
should address this gap.
%Based on the mapping of UBY-LMF to lemon, we performed a conversion of the following
%LRs which demonstrate the main parts of a UBY-LMF -- lemon
%mapping: WordNet, FrameNet, VerbNet, and multilingual Ome\-ga\-Wi\-ki.
%These LRs illustrate a number of differences between the two lexicon models in a
%prototypical way. First, we will describe differences in the representation of senses,
%then we will summarize more general differences that
%had to be harmonized via a mapping. The actual mapping was achieved by means of
%an XML stylesheet transform~\footnote{The XSL file can be downloaded at
%\url{https://raw.github.com/jmccrae/lemon.api/master/src/main/resources/xslt/ubylmf2lemon.xsl}}.

%Although both UBY-LMF and \emph{lemon} are based on LMF,
The XML style sheet mapping between the two lexicon models UBY-LMF and \emph{lemon}
revealed a number of differences between them. Most differences are due to the fact that
\emph{lemon} is a model for ontology lexica where the lexicon and ontology layers are kept separate.
Thus, sense representations in \emph{lemon} primarily consist of references to the associated ontology
where a rich and domain-specific sense definition is provided.
UBY-LMF, on the other hand, represents fine-grained sense information in the lexicon itself.
% CC: moved here from the end, honestly, I did not get the following paragraphs

%The development of UBY-LMF, on the other hand, has been driven by the requirement to cover a large variety of
% lexical information types, which ranges from morphology and lexical syntax to lexical semantics and the mapping
% between syntactic and semantic arguments.
%Thus, the resulting lexicon model makes use of very fine-grained sense specifications which are often grounded in linguistic theories,
%e.g. Frame Semantics (the basis of FrameNet) or the Levin alternation classes of verbs \cite{levin93} (the basis of VerbNet).


%illustrates how the two lexicon models represent senses in different ways:

In line with previous authors~\cite{mccrae2012integrating}, synsets in UBY-LMF (provided in
    WordNet and Ome\-ga\-Wi\-ki) are considered to be SKOS concepts and referenced like
    ontology classes in \emph{lemon}. Hypernym
  relationships stated in the lexicon are mapped directly to broader/narrower
  relationships in the ontology. %In UBY-LMF, Synset is part of the lexicon.
%\item Synsets, which are used in WordNet and Ome\-ga\-Wi\-ki, are considered as
%  entryless senses in \emph{lemon}. Previous
%  conversions~\cite{mccrae2012integrating} have treated synsets as ontological
%  classes, however here the

The ``SemanticPredicate'' class in UBY-LMF is used to represent semantic frames from
  FrameNet \cite{TUD-CS-2013-0003}; frames group senses which evoke the same
  kind of situation with participants taking over particular roles.
  Thus, senses in a FrameNet frame are semantically related, but not synonymous;
  e.g., the verbs {\em love} and {\em hate} are both in the same FrameNet frame. %Experiencer subj frame
In \emph{lemon} this relation is represented as subclass of the general lexical sense and linked to the rest of the lexicon by
a {\em broader} (sense) relationship.

VerbNet verb classes and their hierarchical organization correspond to
  ``Subcat[egorization]FrameSet'' in UBY-LMF. VerbNet classes group verbs that
share the same syntactic subcategorization frame, semantic roles, selectional
restrictions, and semantic predicate. Although the resulting verb classes are
semantically coherent, the semantic relatedness of verb senses in a VerbNet class is much more
distant than in FrameNet, e.g., the verbs {\em believe}, {\em swear} and {\em doubt} are in the same class.
%conjecture-29.5-1.
%semantically related and
%  syntactically similar senses; the semantic relatedness of senses in one VerbNet class is much more
%distant that in FrameNet.
Verb classes in \emph{lemon} are represented by a broader sense as well, so \emph{lemon} collapses the
distinction between FrameNet-style and VerbNet-style sense groupings. 

Translations to other human languages which are encoded as string-valued lexical items in the ``Equivalent'' class
are mapped as a datatype property on the sense, in line with the UBY representation \cite{TC32012} rather than
attempting to construct a cross-lingual linking for these terms as in the original resources
(Ome\-ga\-Wi\-ki and Wiktionary).
Finally, provenance information is added to each resource at the root URL, including the source resource and version, time
of creation and a link to the FOAF profile of the executor. Further examples of usage and modelling are available
from the website~\footnote{e.g., \url{http://lemon-model.net/lexica/uby/modelling.php} or \url{http://lemon-model.net/learn/learn.php}}.
% The hierarchic organisation of VerbNet classes is thus lost, but this can be
% reconstructed by means of an axiomatic description of subcategorization frames in
% order to create a hierarchy similar as in the LexInfo linguistic
% ontology~\cite{mccrae2011linking}.
% \textbf{jmc: Do we need this\ldots I don't really understand it still? Are these SemanticLabels in LMF?}


% Subcategorization frames and senses are not explicitly linked in \emph{lemon} (as in UBY-LMF), but instead
%     are implicitally linked by matching the syntactic and semantic arguments.
%     \textbf{GermaNet is not part of le\-mon\-U\-by}
%     In GermaNet, only
%     syntactic arguments are supplied so we included an explicit link from the sense to its syntactic
%     behaviour.
%\item Syntactic behaviour in UBY-LMF mainly provides information on
%subcategorization frames, which are specified by means of syntactic arguments
%with rich morpho-syntactic information attached
%\cite{TUD-CS-2012-0024}.
%SyntacticBehaviour is merely a property instead of a node in \emph{lemon}: in
%contrast to LMF, there is no need for an explicit link between senses and
%subcategorization frames. Instead this information  can be
%reconstructed from the \emph{synArg} - \emph{semArg} mapping.
%Lexica which do not specify semantic arguments, e.g., GermaNet \cite{Kunze02},
%must provide a specific annotation to allow this information to be represeneted.

%Translations are represented differently. UBY-LMF covers also
%uses ``Equivalent'' for
 % translations that are not sense-disambiguated \cite{TC32012}, where as
 % \emph{lemon} requires that this link is made at the sense level. In the case
 % where it was not possible to infer the sense, the lexical entries a `stub'
 % lexical entry with a single sense.\footnote{UBY-LMF can also represent sense-disambiguated
 %   translations by means of a sense axis which connects senses from different
 % lexica.}
 % \textbf{TODO jmc: We should change the modelling here to perhaps just a simple annotation, so the usage is more practical.
 %   More particularly, should we note that this is due to data lost by UBY-LMF as in the original resources, these
%were links to other lexical entries (as they should be in lemon) but are not represented as such in UBY-LMF?}



% most of the differences between UBY-LMF and \emph{lemon} derive from the fact that
% \emph{lemon} keeps the lexicon and ontology layers separate. In this way, sense representations in
% \emph{lemon} are more compact compared to the more distributed sense representations in UBY-LMF.

%the separation of  the principal distinction between UBY-LMF and \emph{lemon}the separation of
%a principal difference between UBY-LMF and \emph{lemon} is redundant vs. compact  representation of information.
%For instance, the synset class is merged into the lemon:LexicalSense, while UBY might provide information on
%synonyms by sense relations of type synonym and by the Synset class (which groups synonymous senses).
%Since UBY-LMF is populated fully-automatically, the consistency of redundant information can be ensured.
%An example is SyntacticBehaviour which is not used in \emph{lemon}: while the information to which sense a subcat frame belongs to, can be reconstructed from the \emph{synArg} - \emph{semArg} Mapping, the
%relationship between sense and syntactic behaviour can not be recovered for LR which do not specify
%semantic arguments, e.g., GermaNet.




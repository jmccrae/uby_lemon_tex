\section{Large-scale integration of lexical-semantic resources: UBY and UBY-LMF}

%\begin{verbatim}
%JEK: describe uby-lmf lexicon model and uby data
%mapping to lemon -> JEK, JM
%- sense in lemon vs sense in UBY-LMF
%\end{verbatim}
% it is important to keep separate the different levels:
% the standard: LMF (an abstract standard, not directly usable)
% the lexicon model: UBY-LMF vs lemon
% the format: XML vs RDF (vs RelaxNG ...)

\noindent
UBY is both a network of interlinked lexical-semantic resources
and a project on continuous integration and linking of lexical resources for NLP applications.
 It is motivated by the
observation that an essential requirement in NLP is the availability of a wide range of lexical resources that
can be used for many different NLP tasks. In a continuous process, such resources
are integrated into UBY by means of (i) making them interoperable  and (ii)
linking them to other resources in UBY at the sense level.

%UBY has currently integrated 10 LRs in two languages, see www.ukp.tu-darmstadt.de/uby. Only 8 of these LRs have open licenses and
%can be offered as a UBY database dump for download: English WordNet, Wiktionary, Wikipedia, FrameNet and VerbNet, German Wikipedia, Wiktionary, and multilingual OmegaWiki.
%A subset of these LRs is linked at the word sense level and these sense alignments are open as well.
%There are monolingual sense alignments between VerbNet--FrameNet\footnote{\url{http://verbs.colorado.edu/semlink/}} and
%VerbNet--WordNet\footnote{\url{http://verbs.colorado.edu/~mpalmer/projects/verbnet}} as well as between WordNet--Wikipedia \cite{niemann2011peoples} and WordNet--Wiktionary  \cite{meyer2011what}. In addition, Uby provides cross-lingual sense alignments between WordNet and the German OmegaWiki \cite{gurevych2012uby}, also including the inter-language links already given in Wikipedia and OmegaWiki.
%
%UBY databases can be created according to specific application needs: a user might want to use only a subset of the LRs integrated into UBY, convert them to UBY-LMF and import them into
%a database. Any UBY database can be accessed with a single Java-API which is continuously developed along with the conversion tools in an Open Source Project on Google Code (code.google.com/p/uby).
%UBY and UBY-LMF are CC-licensed and the UBY-related software is licensed under the open Apache license.

In UBY, interoperability is achieved by standardizing lexical resources according
to UBY-LMF \cite{ecklekohler2012uby,TUD-CS-2013-0003}, a lexicon
model which is an instantiation of 
    % the ISO standard %% CC: redundant with sect. 1
LMF, specifically designed for NLP.
%instantiates the Lexical Markup Framework (LMF, ISO 24613:2008, \cite{francopoulo2006lexical}).
The lexicon model UBY-LMF has been developed to fully cover a wide range of heterogeneous lexical resources
without information loss, which resulted in a fine-grained model of lexical information types (documented by
data categories from ISOcat,\footnote{\url{http://www.isocat.org/}} 
 the implementation of the ISO 12620:2009 Data Category Registry)
 and was accompanied by
an extension of the ISO standard LMF by a few elements.
The extensibility of UBY-LMF was a primary design principle in order to enable the integration of further 
(in particular automatically acquired) lexical resources. %% , in the future. %% CC: wann sonst ;)

% JEK the use of Data Category Registry such as ISOcat is part of the LMF standard, so it is not necessary to add this here
% (UBY-LMF is an instantiation of LMF), and second, regarding data categories from
% ISOcat -- they are used as comments for li
% it uses externally defined data categories from
% ISOcat as comments.\footnote{\url{http://www.isocat.org/rest/dcs/484}}
%\begin{description}
%\item[Principle of Adoption]: UBY-LMF has been designed to fully cover a wide range of heterogeneous lexical resources
%without information loss.
%This resulted  in a fine-grained model of lexical information types, which ranges from morphology and lexical syntax to lexical semantics and the mapping between syntactic and semantic arguments.
%jmc: Note really sure that this is possible, at any rate lemon covers most things and unlike LMF allows arbitrary extensions

%\item[Independence of implementation]: UBY-LMF  is independent of any particular implementation. There are many ways to implement an LMF lexicon model \cite{francopoulo2007lexical}, including RDF.
% jmc: LMF makes assumptions of an underlying XML model, lemon of an RDF model, both can be mapped to XML, JSON, SQL etc.
%\end{description}


% UBY-LMF has been implemented in two ways: first, as a DTD, and second,
% as a Java Object-Relational Mapping by means of the Hibernate framework\footnote{\url{http://www.hibernate.org}},
% which allows mapping any instance of UBY-LMF either to a SQL database or to an XML file.
%Both ways of representing UBY-LMF do not require the use of globally unique identifiers (URIs).
%However, an implementation of UBY-LMF in RDF would be possible as well.
%For instance, an implementation of LMF to include URIs has been suggested by \cite{francopoulo2007lexical}.
%, since UBY-LMF as such is independent of any particular implementation or
%serialization and an LMF lexicon model can be implemented in many ways \cite{francopoulo2007lexical}.

% JEK I would not call it extension
%An extension of LMF to include URIs \cite{Francopoulo2007}, and full-fledged RDF linearizations of LMF have been suggested, e.g., in the context of the Lexicon Model for Ontologies (Lemon) as described by McCrae et al. (2011)\nocite{McCrae2011}.
% An implementation of LMF to include URIs has already been suggested\cite{francopoulo2007lexical}.
% In fact, providing lexical resources, in particular interlinked resources such as UBY, as linked data is a very natural
% step to take and
% allows us to integrate UBY-LMF-based resources with other resources previously converted to RDF.
%e.g., in the context of the developing Semantic Web.

The mapping from UBY-LMF to \emph{lemon} is motivated by an increase in interoperability with the Semantic Web and its resources, thereby making it available to a new group of potential users and novel applications. %% CC: to motivate the conversion in explicit words
Beyond this, mapping UBY-LMF to \emph{lemon} is an interesting task per se, because \emph{lemon} links lexical resources and
ontologies, whereas UBY-LMF is not related to any ontology. Another benefit is that LMF is not an open standard (in the sense that its specification is not freely available), while \emph{lemon} % provides an interchange format which %% CC: shorter
fully complies with open data and open access principles.

%An extension of LMF to include URIs \cite{Francopoulo2007}, and full-fledged RDF linearizations of LMF have been suggested, e.g., in the context of the Lexicon Model for Ontologies (Lemon) as described by McCrae et al. (2011)\nocite{McCrae2011}.

%Originally, Uby was implemented using the Lexical Markup Framework, a standard aiming for interoperability among lexical-semantic resources.

% LMF is no format, it is an abstract standard
%First, the LMF standard is not an open standard (in the sense that its specification is not freely available), and,
% according to the experience of UBY, the application of the standard requires % making
%NLP domain-specific modifications to the abstract model defined by the standard.
%Second, the currently used implementation of UBY-LMF does not consider how resources can be uniquely identified on the web.


%and in its frequently used
% LMF does not say anything about serialization, XML is just an illustrative example
 %serialization as XML, it does not consider how resources can be uniquely identified on the web.
%Furthermore, according to the experience of UBY, application of the standard requires % making
%NLP domain-specific modifications to the standard schema.

%An RDF formalization tackles %of LMF allows us to tackle
%some of these problems, and this has been suggested by the LMF developers themselves% \citep{francopoulo2009multilingual}
%.\footnote{\url{http://www.tagmatica.fr/lmf/LMF_revision_14_In_OWL29october2007.xml}}

%LMF is grounded in earlier research on feature structures (i.e., directed acyclic graphs) that have been suggested as a generalization over resource-specific data structures \citep{veronis-ide92-feature-structures-for-lexical-dbs}.
%Feature structures are a flexible and general formalism, which became the basis for subsequent standardization, in particular, in LMF.
%\citealp{francopoulo2006lexical}).

% LMF represents a metamodel % aiming to provide a standard
% to represent semantic information in NLP lexicons and machine-readable dictionaries.
% It has been successfully applied to develop resources, including the different Uby data sets.

%Converting a lexical resource in UBY-LMF format to lemon requires a mapping of the UBY-LMF lexicon model to the lemon lexicon model.

% Although both UBY-LMF and \emph{lemon} are based on LMF, the mapping revealed substantial differences which are mainly due to the fact that
% \emph{lemon} is a model for ontology lexica where the lexicon and ontology layers are kept separate. Thus, sense representations in \emph{lemon} primarily consist of references to the associated ontology
% where a rich and domain-specific sense definition is provided.
% The development of UBY-LMF, on the other hand, has been driven by the requirement to cover a large variety of
%  lexical information types, which ranges from morphology and lexical syntax to lexical semantics and the mapping between syntactic and semantic arguments.
% Thus, the resulting lexicon model makes use of very fine-grained sense specifications which are often grounded in linguistic theories,
% e.g. Frame Semantics (the basis of FrameNet) or the Levin alternation classes of verbs \cite{levin93} (the basis of VerbNet).
%


\section{Applications of \emph{le\-mon\-U\-by}}
\noindent
As a web resource, \emph{le\-mon\-U\-by} provides two main possible applications: firstly as a resource
to allow better lexical description of ontology entities and secondly as a resource for NLP applications.

As a wide coverage resource containing lexical information, it would be possible to link existing ontologies
and terminology vocabularies to \emph{le\-mon\-U\-by} and then obtain rich information about the linguistic usage of a given term.
As an example, consider
the cross-lingual linking  of verb senses across English and German available in \emph{le\-mon\-U\-by}. 
Verb senses in the German OmegaWiki can be enriched by semantic role and selectional
preference information from VerbNet and FrameNet
via the cross-lingual linking between German OmegaWiki and WordNet, and via the WordNet--VerbNet and WordNet--FrameNet 
linking.
% based on subcategorization frames.
% The standardized format for English and German subcategorization frames used in \emph{le\-mon\-U\-by} can be exploited for such a
% cross-lingual linking of verb senses. For instance, verb senses from German lexicons
% containing subcategorization frames, such as
% the German wordnet GermaNet \cite{Kunze02} can be linked to resources such as VerbNet, which
% provides access not only to information on semantic roles
% or selectional preferences, but also to semantic role information from FrameNet via the VerbNet--FrameNet
% linking in \emph{le\-mon\-U\-by}. 
Currently, no German lexica with these information types are freely available for
research purposes.

%The OLiA ontologies (and the terminology repositories it is linked with) have been applied as a component of
%the \emph{lemon} model before \cite{mccrae2012interchanging}, and (as part of their original motivation), they can
%be used to represent, compare and integrate linguistic annotations in corpora on the basis of formal concepts rather
%than arbitrary strings \cite{chiarcos2010towards}, and in this function, they also enjoy a certain popularity in NLP
%applications \cite{hellmann2012nif}. In a Linked Data context, they explicitly allow to compare the linguistic categories used in
%Uby with the morphosyntactic annotations in linguistic corpora, if these are represented in RDF. An RDF version of the MASC
%corpus \cite{ide-etal08-masc} has been produced,
%\footnote{
%    The current Linked Data version of the corpus, MASC 1.0.3, generated from data available under \url{http://datahub.io/dataset/masc}, is not yet linked with other resources as it will soon be deprecated by the release of a new version of the corpus and a revision of its data model.
%}
%a resource that also provides FrameNet and WordNet annotations, and whose annotations can thus be directly
%compared with and combined with Uby resources. Information integration between Uby and corpora is another example of
%\emph{le\-mon\-U\-by}.

As a resource for NLP applications, \emph{le\-mon\-U\-by} can be easily deployed into existing
applications~\cite{unger2010generating}. The wide variety of lexical information types it offers, ranging from
taxonomic relationships (e.g., hyponomy) and translations to fine-grained lexical-syntactic information,
makes it an attractive resource for many different NLP tasks, such as Entity or Predicate Disambiguation, to name only two.

Recently, it has been shown that \emph{le\-mon\-U\-by} is a valuable resource for machine translation \cite{mccrae-cimiano:2013:NLP-LOD-SWAIE}. 
By mining translations from \emph{le\-mon\-U\-by} and combining them with
the Moses statistical machine translation, the performance of the resulting machine translation system improved.
% Furthermore, as \emph{le\-mon\-U\-by} is represented in the \emph{lemon} model it can be easily deployed into existing
% NLP applications~\cite{unger2010generating}. 

Furthermore, all the data from \emph{le\-mon\-U\-by} is available via
a SPARQL endpoint.\footnote{\url{http://lemon-model.net/sparql.php}} Thus, ad-hoc queries can be used to access and explore the resource, for example 
to find translational equivalents quickly and easily.

%NLP applications that use \emph{le\-mon\-U\-by} can take advantage of the fact that its component resources are interoperable, both
%structurally and semantically. This enables easy switching between different \emph{le\-mon\-U\-by} lexicons in the context of any
%component-based NLP system.

%Consider as an example of uniform access to \emph{le\-mon\-U\-by} lexicons the following two SPARQL queries that
%retrieve all hyponyms of the lemma {\em tree}, first from WordNet and second from Ome\-ga\-Wi\-ki: {\bf TODO John insert queries}




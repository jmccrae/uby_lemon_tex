\section{Introduction}

\noindent Recently, the language resource community has begun to explore the opportunities offered by the Semantic
Web, lead by the formation of the Linguistic Linked Open Data (LLOD) cloud
and an increasing interest in making use of Linked Open Data principles in the context
of Natural Language Processing (NLP) and Linguistics  \cite{chiarcos2012linked}.
The use of RDF supports data integration and offers a large body of tools for accessing this data.
Furthermore, the linked data approach gives rise to novel research questions in the context of
language resources and their application.

 For lexical resources, data integration has been in
 the focus of interest for many years, resulting in numerous mappings and linkings of lexica, as well as
standards for representing lexical resources, such as the ISO 24613:2008 Lexical Markup Framework
(LMF) \cite{francopoulo2006lexical}. In this context, the LLOD cloud can be considered as a new data integration platform, enabling
linkings not only between lexical resources, but also between lexical resources and
other language resources.

We extend the LLOD cloud by a new lexical resource called \emph{le\-mon\-U\-by}\footnote{\url{http://www.lemon-model.net/lexica/uby/}}
which is
 the result of converting data extracted from the existing large-scale linked lexical resource UBY
 \cite{gurevych2012uby}\footnote{\url{http://www.ukp.tu-darmstadt.de/uby/}} to
 the \emph{lemon} lexicon model.
UBY has been developed independently from Semantic Web technology. It is LMF based
and a subset of the LMF-compliant UBY lexicons is pairwise linked at the word sense level.
The \emph{lemon} lexicon model has been developed for lexical resource integration on the Semantic Web
\cite{mccrae2012interchanging}.
This lexicon model serves as a common interchange format for lexical resources on the Semantic Web
and has been designed to represent and share lexical resources that are linked to ontologies, i.e., ontology lexica.
Making use of a lexicon interchange format, such as \emph{lemon} is not only important for data integration,
but also for the reuse of lexicons.


% A particularly large-scale example of lexical resource integration is the
% lexical-semantic resource UBY \cite{gurevych2012uby}\footnote{\url{http://www.ukp.tu-darmstadt.de/uby/}}.
% UBY has been developed independently from Semantic Web technology and linked data principles.
% UBY is
% \begin{itemize}
%  \item based on LMF, i.e., all lexicons integrated in UBY have been standardized according to the lexicon model
% UBY-LMF \cite{ecklekohler2012uby,TUD-CS-2013-0003}, which is an instantiation of the abstract ISO standard LMF,
% \item a linked lexical resource, i.e., a subset of the UBY lexicons is pairwise linked at the word sense level.
% \end{itemize}
%
% For lexical resource integration on the Semantic Web, the \emph{lemon} lexicon model has been developed \cite{mccrae2012interchanging}.
% This lexicon model serves as a common interchange format for lexical resources on the Semantic Web
% and has been designed to represent and share lexical resources that are linked to ontologies, i.e., ontology lexica.
% Making use of a lexicon interchange format, such as \emph{lemon} is not only important for data integration,
% but also for the reuse of lexicons.

While many lexical resources have already been included in the LLOD cloud, e.g.,
\cite{Bizer_Lehmann_Kobilarov_Auer_Becker_Cyganiak_Hellmann_2009, mccrae2012integrating, navigli2012babelnet, de2008language, deMeloWeikum2008c},
the LLOD cloud is still missing a
large-scale lexical resource rich in lexical information on
verbs, including aspects such as syntactic behaviour and 
semantic roles of a verb's arguments.
%how semantic arguments of verbs can be realised syntactically. 
Such information is crucial for lexicalizing relational knowledge%
%which is often expressed by using verbs %% CC: slightly redundant
, e.g., the relation $like(Experiencer, Theme)$ can be lexicalized syntactically with a verb as in "NP likes NP".

%There has also been some work towards the integration of FrameNet \cite{baker1998berkley} to the Semantic Web \cite{narayanan2003framenet}.
%All these resources provide a substantial body of lexical knowledge, including semantic relations, multilingual
%information and encyclopedic knowledge.

The new resource \emph{le\-mon\-U\-by} addresses this gap: Along with resources for word-level semantics (WordNet \cite{fellbaum98-wordnet}, 
English and German Wiktionary,\footnote{\url{http://www.wiktionary.org}} and the English and German entries of 
Ome\-ga\-Wi\-ki,\footnote{\url{http://www.omegawiki.org}}) we converted two syntactically rich resources from
UBY to the \emph{lemon} format: FrameNet \cite{baker1998berkley} and VerbNet \cite{kipper2008large}. %% CC: list all converted resources here
For further data integration, we established links between \emph{le\-mon\-U\-by} and other language resources in the LLOD cloud.

% We address this gap by converting data from the existing lexical resource UBY to the \emph{lemon} format.
% The following data from UBY were converted: WordNet, FrameNet, VerbNet \cite{kipper2008large},
% English and German Wiktionary~\footnote{\url{http://www.wiktionary.org}}, the English and German
% entries of Ome\-ga\-Wi\-ki~\footnote{\url{http://www.Ome\-ga\-Wi\-ki.org}},
% as well as links between pairs of these lexicons at the word sense level (links between Verb\-Net--Frame\-Net,
% Verb\-Net--Word\-Net, Word\-Net--Frame\-Net, Word\-Net--Wiktionary, Word\-Net -- German Ome\-ga\-Wiki).
% We call the resulting lexical resource \emph{le\-mon\-U\-by}\footnote{\url{http://www.lemon-model.net/lexica/uby/}}
% and established links between \emph{le\-mon\-U\-by} and oher language resources in the LLOD cloud.

% The data-set presented in this paper, \emph{le\-mon\-U\-by}, is the result of
% converting a selection of UBY lexica to
% the \emph{lemon} format:
% it contains interoperable and interlinked versions of WordNet, FrameNet,
% VerbNet \cite{kipper2008large}, English and German Wiktionary~\footnote{\url{http://www.wiktionary.org}},
%  and the English and German entries of
% Ome\-ga\-Wi\-ki~\footnote{\url{http://www.Ome\-ga\-Wi\-ki.org}}.


%Such information is crucial for lexicalizing relational knowledge which is typically expressed
%by using verbs along with specific arguments.


%In the current LLOD cloud, large-scale lexical resources rich in encyclopedic knowledge, such as DBPedia, are
%predominant, while
%the size and diversity of lexical resources rich in linguistic knowledge, particularly for verbs,
% is limited so far.
% There has already been some work towards the integration of FrameNet \cite{baker1998berkley} to the Semantic Web \cite{narayanan2003framenet},
%as well as WordNet
%\cite{van2006conversion}, WordNet and Wiktionary\cite{mccrae2012integrating}, and recently BabelNet as a multilingual
%lexical resource integrating WordNet and Wikipedia \cite{navigli2012babelnet}.
% In the Semantic Web, several versions of WordNet REF
%and FrameNet REF have been available as linked data for quite same time, but due to the LLOD movement
%new resources such as
%Wiktionary REF and BabelNet REF have been added just recently.
%However, the LLOD cloud contains no large-scale lexical resource yet, which is rich in lexical information on
%verbs, including aspects such as their syntactic behavior and how semantic arguments of verbs can be realized
%syntactically. Such information is crucial for lexicalizing relational knowledge which is typically expressed
%by using verbs along with specific arguments.

%  To summarize, our contributions are threefold: (i) an RDF version of data extracted from the existing lexical resource
%  UBY (called \emph{le\-mon\-U\-by}),
% (ii) a mapping of the lexicon model UBY-LMF to \emph{lemon}, and (iii)
% linkings of \emph{le\-mon\-U\-by} to other language resources in the LLOD cloud.

%Moreover, the
%format chosen for the RDF versions of FrameNet and WordNet
% was specific to the underlying data model of these two resources which have been characterized as
%complementary by \cite{Baker:2009}.
%However, the LLOD cloud contains no large-scale lexical resource yet which is rich in lexical information on
%verbs, including aspects such as their syntactic behavior and how semantic arguments of verbs can be realized
%syntactically. Such information is crucial for lexicalizing relational knowledge which is typically expressed
%by using verbs along with specific arguments.
%The dataset \emph{le\-mon\-U\-by} which we present in this paper addresses these gaps:
%it contains interoperable and interlinked versions of WordNet [FEL 98], FrameNet [BAK 98],
%VerbNet [KIP 08], and multilingual Ome\-ga\-Wi\-ki (www.Ome\-ga\-Wi\-ki.org).
%\emph{le\-mon\-U\-by} is linked to the LLOD cloud and has a number of
%additional features that add significant value to the LLOD cloud. We will describe these features and the creation
%of this dataset in the following sections.
